\documentclass[]{article}
\usepackage{lmodern}
\usepackage{amssymb,amsmath}
\usepackage{ifxetex,ifluatex}
\usepackage{fixltx2e} % provides \textsubscript
\ifnum 0\ifxetex 1\fi\ifluatex 1\fi=0 % if pdftex
  \usepackage[T1]{fontenc}
  \usepackage[utf8]{inputenc}
\else % if luatex or xelatex
  \ifxetex
    \usepackage{mathspec}
  \else
    \usepackage{fontspec}
  \fi
  \defaultfontfeatures{Ligatures=TeX,Scale=MatchLowercase}
\fi
% use upquote if available, for straight quotes in verbatim environments
\IfFileExists{upquote.sty}{\usepackage{upquote}}{}
% use microtype if available
\IfFileExists{microtype.sty}{%
\usepackage{microtype}
\UseMicrotypeSet[protrusion]{basicmath} % disable protrusion for tt fonts
}{}
\usepackage[margin=1in]{geometry}
\usepackage{hyperref}
\hypersetup{unicode=true,
            pdftitle={Iris-Classification Project},
            pdfauthor={Henry Chan},
            pdfborder={0 0 0},
            breaklinks=true}
\urlstyle{same}  % don't use monospace font for urls
\usepackage{color}
\usepackage{fancyvrb}
\newcommand{\VerbBar}{|}
\newcommand{\VERB}{\Verb[commandchars=\\\{\}]}
\DefineVerbatimEnvironment{Highlighting}{Verbatim}{commandchars=\\\{\}}
% Add ',fontsize=\small' for more characters per line
\usepackage{framed}
\definecolor{shadecolor}{RGB}{248,248,248}
\newenvironment{Shaded}{\begin{snugshade}}{\end{snugshade}}
\newcommand{\KeywordTok}[1]{\textcolor[rgb]{0.13,0.29,0.53}{\textbf{#1}}}
\newcommand{\DataTypeTok}[1]{\textcolor[rgb]{0.13,0.29,0.53}{#1}}
\newcommand{\DecValTok}[1]{\textcolor[rgb]{0.00,0.00,0.81}{#1}}
\newcommand{\BaseNTok}[1]{\textcolor[rgb]{0.00,0.00,0.81}{#1}}
\newcommand{\FloatTok}[1]{\textcolor[rgb]{0.00,0.00,0.81}{#1}}
\newcommand{\ConstantTok}[1]{\textcolor[rgb]{0.00,0.00,0.00}{#1}}
\newcommand{\CharTok}[1]{\textcolor[rgb]{0.31,0.60,0.02}{#1}}
\newcommand{\SpecialCharTok}[1]{\textcolor[rgb]{0.00,0.00,0.00}{#1}}
\newcommand{\StringTok}[1]{\textcolor[rgb]{0.31,0.60,0.02}{#1}}
\newcommand{\VerbatimStringTok}[1]{\textcolor[rgb]{0.31,0.60,0.02}{#1}}
\newcommand{\SpecialStringTok}[1]{\textcolor[rgb]{0.31,0.60,0.02}{#1}}
\newcommand{\ImportTok}[1]{#1}
\newcommand{\CommentTok}[1]{\textcolor[rgb]{0.56,0.35,0.01}{\textit{#1}}}
\newcommand{\DocumentationTok}[1]{\textcolor[rgb]{0.56,0.35,0.01}{\textbf{\textit{#1}}}}
\newcommand{\AnnotationTok}[1]{\textcolor[rgb]{0.56,0.35,0.01}{\textbf{\textit{#1}}}}
\newcommand{\CommentVarTok}[1]{\textcolor[rgb]{0.56,0.35,0.01}{\textbf{\textit{#1}}}}
\newcommand{\OtherTok}[1]{\textcolor[rgb]{0.56,0.35,0.01}{#1}}
\newcommand{\FunctionTok}[1]{\textcolor[rgb]{0.00,0.00,0.00}{#1}}
\newcommand{\VariableTok}[1]{\textcolor[rgb]{0.00,0.00,0.00}{#1}}
\newcommand{\ControlFlowTok}[1]{\textcolor[rgb]{0.13,0.29,0.53}{\textbf{#1}}}
\newcommand{\OperatorTok}[1]{\textcolor[rgb]{0.81,0.36,0.00}{\textbf{#1}}}
\newcommand{\BuiltInTok}[1]{#1}
\newcommand{\ExtensionTok}[1]{#1}
\newcommand{\PreprocessorTok}[1]{\textcolor[rgb]{0.56,0.35,0.01}{\textit{#1}}}
\newcommand{\AttributeTok}[1]{\textcolor[rgb]{0.77,0.63,0.00}{#1}}
\newcommand{\RegionMarkerTok}[1]{#1}
\newcommand{\InformationTok}[1]{\textcolor[rgb]{0.56,0.35,0.01}{\textbf{\textit{#1}}}}
\newcommand{\WarningTok}[1]{\textcolor[rgb]{0.56,0.35,0.01}{\textbf{\textit{#1}}}}
\newcommand{\AlertTok}[1]{\textcolor[rgb]{0.94,0.16,0.16}{#1}}
\newcommand{\ErrorTok}[1]{\textcolor[rgb]{0.64,0.00,0.00}{\textbf{#1}}}
\newcommand{\NormalTok}[1]{#1}
\usepackage{graphicx,grffile}
\makeatletter
\def\maxwidth{\ifdim\Gin@nat@width>\linewidth\linewidth\else\Gin@nat@width\fi}
\def\maxheight{\ifdim\Gin@nat@height>\textheight\textheight\else\Gin@nat@height\fi}
\makeatother
% Scale images if necessary, so that they will not overflow the page
% margins by default, and it is still possible to overwrite the defaults
% using explicit options in \includegraphics[width, height, ...]{}
\setkeys{Gin}{width=\maxwidth,height=\maxheight,keepaspectratio}
\IfFileExists{parskip.sty}{%
\usepackage{parskip}
}{% else
\setlength{\parindent}{0pt}
\setlength{\parskip}{6pt plus 2pt minus 1pt}
}
\setlength{\emergencystretch}{3em}  % prevent overfull lines
\providecommand{\tightlist}{%
  \setlength{\itemsep}{0pt}\setlength{\parskip}{0pt}}
\setcounter{secnumdepth}{0}
% Redefines (sub)paragraphs to behave more like sections
\ifx\paragraph\undefined\else
\let\oldparagraph\paragraph
\renewcommand{\paragraph}[1]{\oldparagraph{#1}\mbox{}}
\fi
\ifx\subparagraph\undefined\else
\let\oldsubparagraph\subparagraph
\renewcommand{\subparagraph}[1]{\oldsubparagraph{#1}\mbox{}}
\fi

%%% Use protect on footnotes to avoid problems with footnotes in titles
\let\rmarkdownfootnote\footnote%
\def\footnote{\protect\rmarkdownfootnote}

%%% Change title format to be more compact
\usepackage{titling}

% Create subtitle command for use in maketitle
\providecommand{\subtitle}[1]{
  \posttitle{
    \begin{center}\large#1\end{center}
    }
}

\setlength{\droptitle}{-2em}

  \title{Iris-Classification Project}
    \pretitle{\vspace{\droptitle}\centering\huge}
  \posttitle{\par}
    \author{Henry Chan}
    \preauthor{\centering\large\emph}
  \postauthor{\par}
      \predate{\centering\large\emph}
  \postdate{\par}
    \date{January 2, 2020}


\begin{document}
\maketitle

\section{Executive Summary}\label{executive-summary}

The Iris dataset was used in R.A. Fisher's classic 1936 paper, The Use
of Multiple Measurements in Taxonomic Problems, and can also be found on
the UCI Machine Learning Repository.

It includes three iris species with 50 samples each as well as some
properties about each flower. One flower species is linearly separable
from the other two, but the other two are not linearly separable from
each other.

The columns in this dataset are:

1.Id - unique ID of the samples 2.SepalLengthCm - Length of the sepal
(in cm) 3.SepalWidthCm - Width of the sepal (in cm) 4.PetalLengthCm -
Length of the petal (in cm) 5.PetalWidthCm - Width of the petal (in cm)
6.Species - Species name

The aim of the project is to create a machine learning algorithm to
predict the iris species correctly based on the given attributes.

\section{Machine Learning Methods}\label{machine-learning-methods}

\subsubsection{Install Necessary
Packages}\label{install-necessary-packages}

\begin{Shaded}
\begin{Highlighting}[]
\ControlFlowTok{if}\NormalTok{(}\OperatorTok{!}\KeywordTok{require}\NormalTok{(caret)) }\KeywordTok{install.packages}\NormalTok{(}\StringTok{"caret"}\NormalTok{, }\DataTypeTok{repos =} \StringTok{"http://cran.us.r-project.org"}\NormalTok{)}
\end{Highlighting}
\end{Shaded}

\begin{verbatim}
## Loading required package: caret
\end{verbatim}

\begin{verbatim}
## Loading required package: lattice
\end{verbatim}

\begin{verbatim}
## Loading required package: ggplot2
\end{verbatim}

\begin{Shaded}
\begin{Highlighting}[]
\ControlFlowTok{if}\NormalTok{(}\OperatorTok{!}\KeywordTok{require}\NormalTok{(tibble)) }\KeywordTok{install.packages}\NormalTok{(}\StringTok{"tibble"}\NormalTok{, }\DataTypeTok{repos =} \StringTok{"http://cran.us.r-project.org"}\NormalTok{)}
\end{Highlighting}
\end{Shaded}

\begin{verbatim}
## Loading required package: tibble
\end{verbatim}

\begin{Shaded}
\begin{Highlighting}[]
\ControlFlowTok{if}\NormalTok{(}\OperatorTok{!}\KeywordTok{require}\NormalTok{(dplyr)) }\KeywordTok{install.packages}\NormalTok{(}\StringTok{"dplyr"}\NormalTok{, }\DataTypeTok{repos =} \StringTok{"http://cran.us.r-project.org"}\NormalTok{)}
\end{Highlighting}
\end{Shaded}

\begin{verbatim}
## Loading required package: dplyr
\end{verbatim}

\begin{verbatim}
## 
## Attaching package: 'dplyr'
\end{verbatim}

\begin{verbatim}
## The following objects are masked from 'package:stats':
## 
##     filter, lag
\end{verbatim}

\begin{verbatim}
## The following objects are masked from 'package:base':
## 
##     intersect, setdiff, setequal, union
\end{verbatim}

\subsubsection{Load dataset from csv
file}\label{load-dataset-from-csv-file}

\begin{Shaded}
\begin{Highlighting}[]
\NormalTok{data <-}\StringTok{ }\KeywordTok{read.csv}\NormalTok{(}\StringTok{"iris.csv"}\NormalTok{, }\DataTypeTok{header=}\OtherTok{TRUE}\NormalTok{)}
\end{Highlighting}
\end{Shaded}

\subsubsection{Dataset summary}\label{dataset-summary}

\paragraph{Dataset dimensions}\label{dataset-dimensions}

\begin{Shaded}
\begin{Highlighting}[]
\KeywordTok{dim}\NormalTok{(data)}
\end{Highlighting}
\end{Shaded}

\begin{verbatim}
## [1] 150   6
\end{verbatim}

\paragraph{View headers and types of
columns}\label{view-headers-and-types-of-columns}

\begin{Shaded}
\begin{Highlighting}[]
\KeywordTok{sapply}\NormalTok{(data, class)}
\end{Highlighting}
\end{Shaded}

\begin{verbatim}
##            Id SepalLengthCm  SepalWidthCm PetalLengthCm  PetalWidthCm 
##     "integer"     "numeric"     "numeric"     "numeric"     "numeric" 
##       Species 
##      "factor"
\end{verbatim}

\paragraph{List of Species class
levels}\label{list-of-species-class-levels}

\begin{Shaded}
\begin{Highlighting}[]
\KeywordTok{levels}\NormalTok{(data}\OperatorTok{$}\NormalTok{Species)}
\end{Highlighting}
\end{Shaded}

\begin{verbatim}
## [1] "Iris-setosa"     "Iris-versicolor" "Iris-virginica"
\end{verbatim}

\paragraph{Statistcal summary of
dataset}\label{statistcal-summary-of-dataset}

\begin{Shaded}
\begin{Highlighting}[]
\KeywordTok{summary}\NormalTok{(data)}
\end{Highlighting}
\end{Shaded}

\begin{verbatim}
##        Id         SepalLengthCm    SepalWidthCm   PetalLengthCm  
##  Min.   :  1.00   Min.   :4.300   Min.   :2.000   Min.   :1.000  
##  1st Qu.: 38.25   1st Qu.:5.100   1st Qu.:2.800   1st Qu.:1.600  
##  Median : 75.50   Median :5.800   Median :3.000   Median :4.350  
##  Mean   : 75.50   Mean   :5.843   Mean   :3.054   Mean   :3.759  
##  3rd Qu.:112.75   3rd Qu.:6.400   3rd Qu.:3.300   3rd Qu.:5.100  
##  Max.   :150.00   Max.   :7.900   Max.   :4.400   Max.   :6.900  
##   PetalWidthCm              Species  
##  Min.   :0.100   Iris-setosa    :50  
##  1st Qu.:0.300   Iris-versicolor:50  
##  Median :1.300   Iris-virginica :50  
##  Mean   :1.199                       
##  3rd Qu.:1.800                       
##  Max.   :2.500
\end{verbatim}

\paragraph{Distribution of Species by frequency and
percentage}\label{distribution-of-species-by-frequency-and-percentage}

\begin{Shaded}
\begin{Highlighting}[]
\NormalTok{percentage <-}\StringTok{ }\KeywordTok{prop.table}\NormalTok{(}\KeywordTok{table}\NormalTok{(data}\OperatorTok{$}\NormalTok{Species)) }\OperatorTok{*}\StringTok{ }\DecValTok{100}
\KeywordTok{cbind}\NormalTok{(}\DataTypeTok{freq=}\KeywordTok{table}\NormalTok{(data}\OperatorTok{$}\NormalTok{Species), }\DataTypeTok{percentage=}\NormalTok{percentage)}
\end{Highlighting}
\end{Shaded}

\begin{verbatim}
##                 freq percentage
## Iris-setosa       50   33.33333
## Iris-versicolor   50   33.33333
## Iris-virginica    50   33.33333
\end{verbatim}

\paragraph{Split dataset into x and y, y being class
labels}\label{split-dataset-into-x-and-y-y-being-class-labels}

\begin{Shaded}
\begin{Highlighting}[]
\NormalTok{x <-}\StringTok{ }\NormalTok{data[,}\DecValTok{2}\OperatorTok{:}\DecValTok{5}\NormalTok{]}
\NormalTok{y <-}\StringTok{ }\NormalTok{data[,}\DecValTok{6}\NormalTok{]}
\end{Highlighting}
\end{Shaded}

\paragraph{Boxplot for each attribute}\label{boxplot-for-each-attribute}

\includegraphics{Iris-Classification-Project_files/figure-latex/Boxplot for each attribute-1.pdf}
\#\#\#\#Barplot showing frequency of each class
\includegraphics{Iris-Classification-Project_files/figure-latex/Barplot showing frequency of each class-1.pdf}
\#\#\#\#Box and whisker plots by class for each attribute
\includegraphics{Iris-Classification-Project_files/figure-latex/Box and whisker plots by class for each attribute-1.pdf}
\#\#\#\#Density plots by class for each attribute
\includegraphics{Iris-Classification-Project_files/figure-latex/Density plots by class for each attribute-1.pdf}

\subsubsection{Machine Learning Model
Building}\label{machine-learning-model-building}

\paragraph{Split the dataset into training and test set using
createDataPartition(), 80\% of data as training set and 20\% of data as
test
set}\label{split-the-dataset-into-training-and-test-set-using-createdatapartition-80-of-data-as-training-set-and-20-of-data-as-test-set}

\begin{Shaded}
\begin{Highlighting}[]
\NormalTok{test_index <-}\StringTok{ }\KeywordTok{createDataPartition}\NormalTok{(data}\OperatorTok{$}\NormalTok{Species, }\DataTypeTok{p =} \FloatTok{0.8}\NormalTok{, }\DataTypeTok{list =} \OtherTok{FALSE}\NormalTok{)}
\NormalTok{train <-}\StringTok{ }\NormalTok{data[}\OperatorTok{-}\NormalTok{test_index,]}
\NormalTok{test <-}\StringTok{ }\NormalTok{data[test_index,]}
\end{Highlighting}
\end{Shaded}

\paragraph{Algorithms will be assessed using 10-fold crossvalidation,
setup
here}\label{algorithms-will-be-assessed-using-10-fold-crossvalidation-setup-here}

\begin{Shaded}
\begin{Highlighting}[]
\NormalTok{control <-}\StringTok{ }\KeywordTok{trainControl}\NormalTok{(}\DataTypeTok{method=}\StringTok{"cv"}\NormalTok{, }\DataTypeTok{number=}\DecValTok{10}\NormalTok{)}
\NormalTok{metric <-}\StringTok{ "Accuracy"}
\end{Highlighting}
\end{Shaded}

\paragraph{5 machine learning models are introduced and respective
accuracy of the prediction on test set are
compared}\label{machine-learning-models-are-introduced-and-respective-accuracy-of-the-prediction-on-test-set-are-compared}

Linear Discriminant Analysis

\begin{Shaded}
\begin{Highlighting}[]
\KeywordTok{set.seed}\NormalTok{(}\DecValTok{1}\NormalTok{)}
\NormalTok{fit.lda <-}\StringTok{ }\KeywordTok{train}\NormalTok{(Species}\OperatorTok{~}\NormalTok{., }\DataTypeTok{data=}\NormalTok{data, }\DataTypeTok{method=}\StringTok{"lda"}\NormalTok{, }\DataTypeTok{metric=}\NormalTok{metric, }\DataTypeTok{trControl=}\NormalTok{control)}
\NormalTok{predictions.lda <-}\StringTok{ }\KeywordTok{predict}\NormalTok{(fit.lda,test)}
\end{Highlighting}
\end{Shaded}

Decision Tree

\begin{Shaded}
\begin{Highlighting}[]
\KeywordTok{set.seed}\NormalTok{(}\DecValTok{1}\NormalTok{)}
\NormalTok{fit.rpart <-}\StringTok{ }\KeywordTok{train}\NormalTok{(Species}\OperatorTok{~}\NormalTok{., }\DataTypeTok{data=}\NormalTok{data, }\DataTypeTok{method=}\StringTok{"rpart"}\NormalTok{, }\DataTypeTok{metric=}\NormalTok{metric, }\DataTypeTok{trControl=}\NormalTok{control)}
\NormalTok{predictions.rpart <-}\StringTok{ }\KeywordTok{predict}\NormalTok{(fit.rpart,test)}
\end{Highlighting}
\end{Shaded}

k-Nearest Neighbors

\begin{Shaded}
\begin{Highlighting}[]
\KeywordTok{set.seed}\NormalTok{(}\DecValTok{1}\NormalTok{)}
\NormalTok{fit.knn <-}\StringTok{ }\KeywordTok{train}\NormalTok{(Species}\OperatorTok{~}\NormalTok{., }\DataTypeTok{data=}\NormalTok{data, }\DataTypeTok{method=}\StringTok{"knn"}\NormalTok{, }\DataTypeTok{metric=}\NormalTok{metric, }\DataTypeTok{trControl=}\NormalTok{control)}
\NormalTok{predictions.knn <-}\StringTok{ }\KeywordTok{predict}\NormalTok{(fit.knn,test)}
\end{Highlighting}
\end{Shaded}

Support Vector Machines

\begin{Shaded}
\begin{Highlighting}[]
\KeywordTok{set.seed}\NormalTok{(}\DecValTok{1}\NormalTok{)}
\NormalTok{fit.svm <-}\StringTok{ }\KeywordTok{train}\NormalTok{(Species}\OperatorTok{~}\NormalTok{., }\DataTypeTok{data=}\NormalTok{data, }\DataTypeTok{method=}\StringTok{"svmRadial"}\NormalTok{, }\DataTypeTok{metric=}\NormalTok{metric, }\DataTypeTok{trControl=}\NormalTok{control)}
\NormalTok{predictions.svm <-}\StringTok{ }\KeywordTok{predict}\NormalTok{(fit.svm,test)}
\end{Highlighting}
\end{Shaded}

Random Forest

\begin{Shaded}
\begin{Highlighting}[]
\KeywordTok{set.seed}\NormalTok{(}\DecValTok{1}\NormalTok{)}
\NormalTok{fit.rf <-}\StringTok{ }\KeywordTok{train}\NormalTok{(Species}\OperatorTok{~}\NormalTok{., }\DataTypeTok{data=}\NormalTok{data, }\DataTypeTok{method=}\StringTok{"rf"}\NormalTok{, }\DataTypeTok{metric=}\NormalTok{metric, }\DataTypeTok{trControl=}\NormalTok{control)}
\NormalTok{predictions.rf <-}\StringTok{ }\KeywordTok{predict}\NormalTok{(fit.rf,test)}
\end{Highlighting}
\end{Shaded}

\section{Results}\label{results}

\subsubsection{Summarize model
accuracies}\label{summarize-model-accuracies}

\paragraph{Summary of Accuracy and Kappa of different
models}\label{summary-of-accuracy-and-kappa-of-different-models}

\begin{Shaded}
\begin{Highlighting}[]
\NormalTok{results <-}\StringTok{ }\KeywordTok{resamples}\NormalTok{(}\KeywordTok{list}\NormalTok{(}\DataTypeTok{lda=}\NormalTok{fit.lda, }\DataTypeTok{cart=}\NormalTok{fit.rpart, }\DataTypeTok{knn=}\NormalTok{fit.knn, }\DataTypeTok{svm=}\NormalTok{fit.svm, }\DataTypeTok{rf=}\NormalTok{fit.rf))}
\KeywordTok{summary}\NormalTok{(results)}
\end{Highlighting}
\end{Shaded}

\begin{verbatim}
## 
## Call:
## summary.resamples(object = results)
## 
## Models: lda, cart, knn, svm, rf 
## Number of resamples: 10 
## 
## Accuracy 
##           Min. 1st Qu. Median      Mean 3rd Qu. Max. NA's
## lda  1.0000000       1      1 1.0000000       1    1    0
## cart 0.9333333       1      1 0.9866667       1    1    0
## knn  1.0000000       1      1 1.0000000       1    1    0
## svm  0.9333333       1      1 0.9933333       1    1    0
## rf   1.0000000       1      1 1.0000000       1    1    0
## 
## Kappa 
##      Min. 1st Qu. Median Mean 3rd Qu. Max. NA's
## lda   1.0       1      1 1.00       1    1    0
## cart  0.9       1      1 0.98       1    1    0
## knn   1.0       1      1 1.00       1    1    0
## svm   0.9       1      1 0.99       1    1    0
## rf    1.0       1      1 1.00       1    1    0
\end{verbatim}

\begin{Shaded}
\begin{Highlighting}[]
\KeywordTok{dotplot}\NormalTok{(results)}
\end{Highlighting}
\end{Shaded}

\includegraphics{Iris-Classification-Project_files/figure-latex/accuracy and kappa of models-1.pdf}
\#\#\#\#Evaluate confustion matrix of the models' predictions on test
data

\begin{Shaded}
\begin{Highlighting}[]
\KeywordTok{confusionMatrix}\NormalTok{(predictions.lda, test}\OperatorTok{$}\NormalTok{Species)}
\end{Highlighting}
\end{Shaded}

\begin{verbatim}
## Confusion Matrix and Statistics
## 
##                  Reference
## Prediction        Iris-setosa Iris-versicolor Iris-virginica
##   Iris-setosa              40               0              0
##   Iris-versicolor           0              40              0
##   Iris-virginica            0               0             40
## 
## Overall Statistics
##                                      
##                Accuracy : 1          
##                  95% CI : (0.9697, 1)
##     No Information Rate : 0.3333     
##     P-Value [Acc > NIR] : < 2.2e-16  
##                                      
##                   Kappa : 1          
##                                      
##  Mcnemar's Test P-Value : NA         
## 
## Statistics by Class:
## 
##                      Class: Iris-setosa Class: Iris-versicolor
## Sensitivity                      1.0000                 1.0000
## Specificity                      1.0000                 1.0000
## Pos Pred Value                   1.0000                 1.0000
## Neg Pred Value                   1.0000                 1.0000
## Prevalence                       0.3333                 0.3333
## Detection Rate                   0.3333                 0.3333
## Detection Prevalence             0.3333                 0.3333
## Balanced Accuracy                1.0000                 1.0000
##                      Class: Iris-virginica
## Sensitivity                         1.0000
## Specificity                         1.0000
## Pos Pred Value                      1.0000
## Neg Pred Value                      1.0000
## Prevalence                          0.3333
## Detection Rate                      0.3333
## Detection Prevalence                0.3333
## Balanced Accuracy                   1.0000
\end{verbatim}

\begin{Shaded}
\begin{Highlighting}[]
\KeywordTok{confusionMatrix}\NormalTok{(predictions.rpart, test}\OperatorTok{$}\NormalTok{Species)}
\end{Highlighting}
\end{Shaded}

\begin{verbatim}
## Confusion Matrix and Statistics
## 
##                  Reference
## Prediction        Iris-setosa Iris-versicolor Iris-virginica
##   Iris-setosa              40               0              0
##   Iris-versicolor           0              40              0
##   Iris-virginica            0               0             40
## 
## Overall Statistics
##                                      
##                Accuracy : 1          
##                  95% CI : (0.9697, 1)
##     No Information Rate : 0.3333     
##     P-Value [Acc > NIR] : < 2.2e-16  
##                                      
##                   Kappa : 1          
##                                      
##  Mcnemar's Test P-Value : NA         
## 
## Statistics by Class:
## 
##                      Class: Iris-setosa Class: Iris-versicolor
## Sensitivity                      1.0000                 1.0000
## Specificity                      1.0000                 1.0000
## Pos Pred Value                   1.0000                 1.0000
## Neg Pred Value                   1.0000                 1.0000
## Prevalence                       0.3333                 0.3333
## Detection Rate                   0.3333                 0.3333
## Detection Prevalence             0.3333                 0.3333
## Balanced Accuracy                1.0000                 1.0000
##                      Class: Iris-virginica
## Sensitivity                         1.0000
## Specificity                         1.0000
## Pos Pred Value                      1.0000
## Neg Pred Value                      1.0000
## Prevalence                          0.3333
## Detection Rate                      0.3333
## Detection Prevalence                0.3333
## Balanced Accuracy                   1.0000
\end{verbatim}

\begin{Shaded}
\begin{Highlighting}[]
\KeywordTok{confusionMatrix}\NormalTok{(predictions.knn, test}\OperatorTok{$}\NormalTok{Species)}
\end{Highlighting}
\end{Shaded}

\begin{verbatim}
## Confusion Matrix and Statistics
## 
##                  Reference
## Prediction        Iris-setosa Iris-versicolor Iris-virginica
##   Iris-setosa              40               0              0
##   Iris-versicolor           0              40              0
##   Iris-virginica            0               0             40
## 
## Overall Statistics
##                                      
##                Accuracy : 1          
##                  95% CI : (0.9697, 1)
##     No Information Rate : 0.3333     
##     P-Value [Acc > NIR] : < 2.2e-16  
##                                      
##                   Kappa : 1          
##                                      
##  Mcnemar's Test P-Value : NA         
## 
## Statistics by Class:
## 
##                      Class: Iris-setosa Class: Iris-versicolor
## Sensitivity                      1.0000                 1.0000
## Specificity                      1.0000                 1.0000
## Pos Pred Value                   1.0000                 1.0000
## Neg Pred Value                   1.0000                 1.0000
## Prevalence                       0.3333                 0.3333
## Detection Rate                   0.3333                 0.3333
## Detection Prevalence             0.3333                 0.3333
## Balanced Accuracy                1.0000                 1.0000
##                      Class: Iris-virginica
## Sensitivity                         1.0000
## Specificity                         1.0000
## Pos Pred Value                      1.0000
## Neg Pred Value                      1.0000
## Prevalence                          0.3333
## Detection Rate                      0.3333
## Detection Prevalence                0.3333
## Balanced Accuracy                   1.0000
\end{verbatim}

\begin{Shaded}
\begin{Highlighting}[]
\KeywordTok{confusionMatrix}\NormalTok{(predictions.svm, test}\OperatorTok{$}\NormalTok{Species)}
\end{Highlighting}
\end{Shaded}

\begin{verbatim}
## Confusion Matrix and Statistics
## 
##                  Reference
## Prediction        Iris-setosa Iris-versicolor Iris-virginica
##   Iris-setosa              40               0              0
##   Iris-versicolor           0              40              1
##   Iris-virginica            0               0             39
## 
## Overall Statistics
##                                           
##                Accuracy : 0.9917          
##                  95% CI : (0.9544, 0.9998)
##     No Information Rate : 0.3333          
##     P-Value [Acc > NIR] : < 2.2e-16       
##                                           
##                   Kappa : 0.9875          
##                                           
##  Mcnemar's Test P-Value : NA              
## 
## Statistics by Class:
## 
##                      Class: Iris-setosa Class: Iris-versicolor
## Sensitivity                      1.0000                 1.0000
## Specificity                      1.0000                 0.9875
## Pos Pred Value                   1.0000                 0.9756
## Neg Pred Value                   1.0000                 1.0000
## Prevalence                       0.3333                 0.3333
## Detection Rate                   0.3333                 0.3333
## Detection Prevalence             0.3333                 0.3417
## Balanced Accuracy                1.0000                 0.9938
##                      Class: Iris-virginica
## Sensitivity                         0.9750
## Specificity                         1.0000
## Pos Pred Value                      1.0000
## Neg Pred Value                      0.9877
## Prevalence                          0.3333
## Detection Rate                      0.3250
## Detection Prevalence                0.3250
## Balanced Accuracy                   0.9875
\end{verbatim}

\begin{Shaded}
\begin{Highlighting}[]
\KeywordTok{confusionMatrix}\NormalTok{(predictions.rf, test}\OperatorTok{$}\NormalTok{Species)}
\end{Highlighting}
\end{Shaded}

\begin{verbatim}
## Confusion Matrix and Statistics
## 
##                  Reference
## Prediction        Iris-setosa Iris-versicolor Iris-virginica
##   Iris-setosa              40               0              0
##   Iris-versicolor           0              40              0
##   Iris-virginica            0               0             40
## 
## Overall Statistics
##                                      
##                Accuracy : 1          
##                  95% CI : (0.9697, 1)
##     No Information Rate : 0.3333     
##     P-Value [Acc > NIR] : < 2.2e-16  
##                                      
##                   Kappa : 1          
##                                      
##  Mcnemar's Test P-Value : NA         
## 
## Statistics by Class:
## 
##                      Class: Iris-setosa Class: Iris-versicolor
## Sensitivity                      1.0000                 1.0000
## Specificity                      1.0000                 1.0000
## Pos Pred Value                   1.0000                 1.0000
## Neg Pred Value                   1.0000                 1.0000
## Prevalence                       0.3333                 0.3333
## Detection Rate                   0.3333                 0.3333
## Detection Prevalence             0.3333                 0.3333
## Balanced Accuracy                1.0000                 1.0000
##                      Class: Iris-virginica
## Sensitivity                         1.0000
## Specificity                         1.0000
## Pos Pred Value                      1.0000
## Neg Pred Value                      1.0000
## Prevalence                          0.3333
## Detection Rate                      0.3333
## Detection Prevalence                0.3333
## Balanced Accuracy                   1.0000
\end{verbatim}

\subsubsection{Create a table to summarise the accuracy of different
models}\label{create-a-table-to-summarise-the-accuracy-of-different-models}

Linear Discriminant Analysis

\begin{Shaded}
\begin{Highlighting}[]
\NormalTok{cm <-}\StringTok{ }\KeywordTok{confusionMatrix}\NormalTok{(predictions.lda, test}\OperatorTok{$}\NormalTok{Species)}
\NormalTok{overall <-}\StringTok{ }\NormalTok{cm}\OperatorTok{$}\NormalTok{overall}
\NormalTok{overall.accuracy <-}\StringTok{ }\NormalTok{overall[}\StringTok{'Accuracy'}\NormalTok{] }

\NormalTok{Summary <-}\StringTok{ }\KeywordTok{tibble}\NormalTok{(}\DataTypeTok{Model =} \StringTok{"lda"}\NormalTok{, }\DataTypeTok{Accuracy =}\NormalTok{ overall.accuracy)}
\end{Highlighting}
\end{Shaded}

Decision Tree

\begin{Shaded}
\begin{Highlighting}[]
\NormalTok{cm <-}\StringTok{ }\KeywordTok{confusionMatrix}\NormalTok{(predictions.rpart, test}\OperatorTok{$}\NormalTok{Species)}
\NormalTok{overall <-}\StringTok{ }\NormalTok{cm}\OperatorTok{$}\NormalTok{overall}
\NormalTok{overall.accuracy <-}\StringTok{ }\NormalTok{overall[}\StringTok{'Accuracy'}\NormalTok{] }

\NormalTok{Summary <-}\StringTok{ }\KeywordTok{bind_rows}\NormalTok{(Summary, }
                     \KeywordTok{tibble}\NormalTok{(}\DataTypeTok{Model=}\StringTok{"rpart"}\NormalTok{,}
                            \DataTypeTok{Accuracy =}\NormalTok{ overall.accuracy))}
\end{Highlighting}
\end{Shaded}

k-Nearest Neighbors

\begin{Shaded}
\begin{Highlighting}[]
\NormalTok{cm <-}\StringTok{ }\KeywordTok{confusionMatrix}\NormalTok{(predictions.knn, test}\OperatorTok{$}\NormalTok{Species)}
\NormalTok{overall <-}\StringTok{ }\NormalTok{cm}\OperatorTok{$}\NormalTok{overall}
\NormalTok{overall.accuracy <-}\StringTok{ }\NormalTok{overall[}\StringTok{'Accuracy'}\NormalTok{] }

\NormalTok{Summary <-}\StringTok{ }\KeywordTok{bind_rows}\NormalTok{(Summary, }
                     \KeywordTok{tibble}\NormalTok{(}\DataTypeTok{Model=}\StringTok{"knn"}\NormalTok{,}
                            \DataTypeTok{Accuracy =}\NormalTok{ overall.accuracy))}
\end{Highlighting}
\end{Shaded}

Support Vector Machines

\begin{Shaded}
\begin{Highlighting}[]
\NormalTok{cm <-}\StringTok{ }\KeywordTok{confusionMatrix}\NormalTok{(predictions.svm, test}\OperatorTok{$}\NormalTok{Species)}
\NormalTok{overall <-}\StringTok{ }\NormalTok{cm}\OperatorTok{$}\NormalTok{overall}
\NormalTok{overall.accuracy <-}\StringTok{ }\NormalTok{overall[}\StringTok{'Accuracy'}\NormalTok{] }

\NormalTok{Summary <-}\StringTok{ }\KeywordTok{bind_rows}\NormalTok{(Summary, }
                     \KeywordTok{tibble}\NormalTok{(}\DataTypeTok{Model=}\StringTok{"svm"}\NormalTok{,}
                            \DataTypeTok{Accuracy =}\NormalTok{ overall.accuracy))}
\end{Highlighting}
\end{Shaded}

Random Forest

\begin{Shaded}
\begin{Highlighting}[]
\NormalTok{cm <-}\StringTok{ }\KeywordTok{confusionMatrix}\NormalTok{(predictions.rf, test}\OperatorTok{$}\NormalTok{Species)}
\NormalTok{overall <-}\StringTok{ }\NormalTok{cm}\OperatorTok{$}\NormalTok{overall}
\NormalTok{overall.accuracy <-}\StringTok{ }\NormalTok{overall[}\StringTok{'Accuracy'}\NormalTok{] }

\NormalTok{Summary <-}\StringTok{ }\KeywordTok{bind_rows}\NormalTok{(Summary, }
                     \KeywordTok{tibble}\NormalTok{(}\DataTypeTok{Model=}\StringTok{"rf"}\NormalTok{,}
                            \DataTypeTok{Accuracy =}\NormalTok{ overall.accuracy))}
\end{Highlighting}
\end{Shaded}

\subsubsection{Print summary table of models'
accuracy}\label{print-summary-table-of-models-accuracy}

\begin{Shaded}
\begin{Highlighting}[]
\KeywordTok{print}\NormalTok{(Summary)}
\end{Highlighting}
\end{Shaded}

\begin{verbatim}
## # A tibble: 5 x 2
##   Model Accuracy
##   <chr>    <dbl>
## 1 lda      1    
## 2 rpart    1    
## 3 knn      1    
## 4 svm      0.992
## 5 rf       1
\end{verbatim}

\section{Conclusion}\label{conclusion}

In the project, 5 models are introduced: Linear Discriminant Analysis
(lda), Decision Tree (rpart), k-Nearest Neighbors (knn), Support Vector
Machines (svm), Random Forest (rf). Algorithms are built based on train
set data and are applied to test set for prediction. Accuracy of
predictions of different models is summarised in the table. Based on the
result, all models give 100\% accuracy.


\end{document}
